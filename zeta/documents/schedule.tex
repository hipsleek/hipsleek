\documentclass{article}

\title{Research Schedule}
\author{Vu An Hoa}

\begin{document}

\maketitle

\begin{abstract}
\begin{scriptsize}
This document drafts my research and implementation plan. The goal is to obtain a working number theoretic theorem prover that supports: (i) Handling of elementary algebraic problems concerning of integers such as divisibility, gcd, Fibonacci, factorial, etc.; (ii) Automated induction using user annotation. We expect to solve the current set of theorems. Currently, we already got a working parser, type inference, standard translation to Z3 that should work perfectly for propositional properties.
\end{scriptsize}
\end{abstract}

\begin{enumerate}

\item [12-16 March]

\begin{enumerate}
\item Finish the summaries of classical articles in automated theorem proving.
\item Explore some systems introduced in CADE ATP System Competition (CASC) like \texttt{Vampire}, \texttt{iProver}, \texttt{Paradox}.
\item Take a look at the CASC problems.
\item Draft the implementation notes and plan
\item Refactor existing implementation
\begin{itemize}
\item Use generic type for \textbf{term} term data structure to allow the use of different underlying types for variable and numerical constants.\\
Currently, we are using \textbf{string} to name variables. It is necessary for parsing the input files, but not for efficient manipulation in later stages. Translate them to \textbf{int} will be much more convenient.
\item Remove the sort information from the variable. Formulas should have associated variables to sort.
\item Redesign internal data structures using records and getter.
\end{itemize}
\item Enhance current translation to Z3. Sorts, variables, functions, etc. should be computed once and retrieve later. Distinguish arrays and functions.
\item Finish Skolemization of formulas.
\end{enumerate}

\item [19-23 March]

\begin{enumerate}
\item Finish Skolemization of formulas.
\item Normalize algebraic  terms
\item Implements simple equation solver
\item Utilize \texttt{Reduce} as an underlying equation solver. Note that this requires generating input, parsing output, etc. (Can reuse the code in hip/sleek \texttt{redlog} module.)
\item 
\item TO BE FILL UP AT THE END OF PREVIOUS WEEK
\end{enumerate}

\item [26-30 March]
\begin{enumerate}
\item TO BE FILL UP AT THE END OF PREVIOUS WEEK
\end{enumerate}

\item [2-6 April]
\begin{enumerate}
\item TO BE FILL UP AT THE END OF PREVIOUS WEEK
\end{enumerate}

\item [9-13 April]
\begin{enumerate}
\item TO BE FILL UP AT THE END OF PREVIOUS WEEK
\end{enumerate}

\item [16-20 April]
\begin{enumerate}
\item TO BE FILL UP AT THE END OF PREVIOUS WEEK
\end{enumerate}

\end{enumerate}

\end{document}