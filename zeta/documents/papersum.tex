\documentclass{article}
\usepackage{amsmath,amsthm,amssymb}

\begin{document}
\title{Brief summaries for articles in \textsc{Automation of Reasoning: Classical Papers on Computational Logic, Volume 1}}
\author{Vu An Hoa}
\maketitle

\section{A. Newell, J.C. Shaw, H. A. Simon - \emph{Logic Theory Machine}}

\begin{itemize}
\item Operates on heuristics
\item Prove theorems of sentential calculus
\item Generate the proof sequence backward (starting from target theorem, find the hypotheses)
\item Three methods (rules):
\begin{itemize}
\item Substitution: find instances of axioms or proved theorems in the database to be used in the proof
\item Detachment: generate sub-goals from MP, to prove $\varphi$, find a probable $\psi$ and then proceed to justify two subgoals
$$\psi$$
and
$$\psi \rightarrow \varphi.$$
\item Chaining: similar to detachment but work on transitivity of implication i.e. justify $\alpha \rightarrow \beta$ by guessing an $\gamma$ such that
$$\alpha \rightarrow \gamma$$
and
$$\gamma \rightarrow \beta$$
are provable.
\end{itemize}
In guessing of subproblems methods, the choice is typically made so that one of subproblems automatically follows the axioms or proved theorems.
\item Solve 38/52 problems in \emph{Principia Mathematica}
\end{itemize}

\section{E. Beth's \emph{Semantics Tableaux}}

\begin{itemize}
\item Truth table analysis
\item Possible to convert to classical proofs
\item 
\end{itemize}

\section{A. Robinson - \emph{Herbrand universe}}

\begin{itemize}
\item Equisatisfiability of a formula and its Skolemized form
\item Herbrand's theorem (followed either Compactness theorem or G\"odel completeness theorem): Unsatisfiability of universal formula reduces to unsatisfiability of a finite conjunction. (A dual version: An formula is provable if and only if a \emph{finite} disjunction of its Herbrandization instantiated by terms made of Herbrand functions (analogous to Skolem function) and constants is provable.

For example: $\forall x F(x)$ where $F$ has no quantifier is unsatisfiable if and only if for a finite number of term $t_1, t_2, ..., t_k \in \{a, b, ..., \}$, the collection $\{F(t_1), F(t_2), ..., F(t_k)\}$ is already unsatisfiable.
\item The collection of terms generated by Skolem functions and all constants is now referred to as \emph{Herbrand's universe}  (it should better be called \emph{Skolem universe}).
\item Observation that Gelernter's geometry theorem prover produces additional points, lines, etc. which are simply elements of Herbrand's universe.
\item Robinson's suggestions:
\begin{itemize}
\item Find an efficient method to select individuals in Herbrand's universe using ``mathematical insights'' as in Gelernter geometry theorem prover
\item Generate new mathematical concepts from Skolemization
\end{itemize}
\end{itemize}

\section{H. Wang - \emph{Toward mechanical mathematics}}

\begin{itemize}
\item Advocate ``inferential'' analysis as an analogous tool to computer numerical computations
\item Three programs
\begin{itemize}
\item Propositional logic theorem prover: depth first search approach, equivalent to truth table analysis
\item Propositional logic theorem constructor: generate terms and filter ``non-trivial'' theorems
\item $\forall \exists$ first-order theorem prover: solve theorems of form $\forall^{*} \exists^{*}$ in first-order logic by: transform to prenex normal form, substitute by variables and check for propositional implication
\end{itemize}
\end{itemize}

\section{D. Prawitz - \emph{Improved proof procedure}}

\section{M. Davis, H. Putnam - \emph{Procedure for quantification theory}}

\begin{itemize}
\item Provide heuristic for better (un)satisfiability checking of a collection $\Gamma$ in sentential logic.
\begin{itemize}
\item Rule 0: If $\Gamma$ is empty then it is consistent.
\item Rule I (Elimination of one literal clause): basically, if $p$ is has a single choice to ensure satisfiability of $\Gamma$ (which typically happens when $p$ or $\neg p$ stands alone in $\Gamma$), it must take that choice.
\begin{itemize}
\item $\{p,\neg p\} \vdash \bot$
\item For any literal $p \in \Gamma$, delete $p$ in conjunction and $\neg p$ in disjunction.
\item By symmetry, if $\neg p \in \Gamma$, delete $\neg p$ in conjunction and $p$ in disjunction.
\end{itemize}
\item Rule II (Affirmative-negative): If only $p$ (or $\neg p$) occurs in $\Gamma$, simply set it true and remove all occurrences.
\item Rule III (Elimitation of atomic formulas): Replace $$(\alpha \lor p) \land (\beta \lor \neg p) \land \gamma$$ by $$(\alpha \lor \beta) \land \gamma.$$
\end{itemize}
\item In a later (implementation) paper \emph{A machine program for theorem proving}, rule III is replaced by a split rule III* which splits the satisfiability of $$(\alpha \lor \beta) \land \gamma$$ of rule III to satisfiability of BOTH
$$\alpha \land \gamma \quad \text{ and } \quad \beta \land \gamma.$$
\item Utilizes Robinson's Herbrand universe idea, obtain a semi-decidable algorithm for first order logic.
\item \textbf{Note:} I realize that transformation to PNF is unnecessary and one might perform \emph{local Skolemization} instead. Moreover, quantification are typically and purposely put at the place where it is necessary. So the transformation to PNF is good for programming purpose; but it destroys the mathematical intuition of the quantification.
\end{itemize}

\section{J.A. Robinson -  \emph{Resolution principle}}

\begin{itemize}
\item 	
\end{itemize}

\end{document}